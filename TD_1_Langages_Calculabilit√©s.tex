\documentclass[a4paper,11pt]{article}%

%%%%%%%%%%%%%%%%%%%%%%%%%%%%%%%%%%%%%%%%%%%%%%%%%%%%%%%%%%%%%%%%%%%%%%%%%%%%%%%%%%%%%%%%%%%%%%

%Global structure parameters

\usepackage{fullpage}%

\usepackage[francais]{babel}%

\usepackage[utf8]{inputenc}%
\usepackage[T1]{fontenc}%

%Font selection : http://www.tux.dk/FontCatalogue/newpx

%\usepackage{newpxtext}%
%\usepackage{newpxmath}%

%Macro packages

\usepackage{url}%
\usepackage{graphicx}%
\usepackage{listings}%

%Parameters for listings

\lstset{%
	basicstyle=\footnosize\sffamily,%
	columns=fullflexible,%
	frame=lb,%
	frameround=fftf,%
	language=caml,%
}%

%Fine tuning

\setlength{\parskip}{0.5\baselineskip}%

%%%%%%%%%%%%%%%%%%%%%%%%%%%%%%%%%%%%%%%%%%%%%%%%%%%%%%%%%%%%%%%%%%%%%%%%%%%%%%%%%%%%%%%%%%%%%%

\begin{document}

\title{Correction TD 1 : Expressions rationnelles et automates finis}

\author{Tom Bachard, Antoine Gonon}

\date{15 septembre 2017}

\maketitle


\section*{\bf{Exercice 1 }}%
\it
De l'expression rationnelle à l'automate: algorithme de Thompson
\\
\\
\rm
Voici l'automate reconnaissant le langage dénoté par l'expression rationnelle
\it
$a^*b+(aba)^*$
\rm
: %

\section*{\bf{Exercice 2 }}%
\it
De l'automate à l'expression rationnelle: équations et lemme d'Arden
\rm{Soit $A=(Q, \Sigma, \delta, q_0, F)$ un automate fini.}
\\
\subsection{}
Pour tout q\in \Q, on considère $L_e=\{w\in \Sigma^*, \delta^*(e,w)\in F\}$, l'ensemble des mots reconnus par l'automate (Q, \Sigma, \delta, e, F). En particulier, $L_{q_0}$ est le langage reconnu par l'automate A.
\\
On obtient le système suivant d'équations.
\\
$L_0=(a+b)L_0+aL_1$ (1)\\
$L_1=aL_0+bL_2$ (2)\\
$L_2=(a+b)L_2+aL_1+\epsilon$ (3)\\

\subsection{}
On résout (1) et (3) grâce au lemme d'Arden, les conditions pour l'unicité étant vérifiées, on obtient: %
\\
$L_0=(a+b)^*aL_1$\\
$L_2=(a+b)^*(aL_1+\epsilon )$\\
On réinjecte $L_0$ et $L_2$ dans (2). On peut alors utiliser le lemme d'Arden pour trouver $L_1$, les conditions pour l'unicité étant vérifiées, on obtient:\\
$L_1=((a+b)(a+b)^*a)^*b(a+b)^*$\\
Ce qui nous permet d'obtenir $L_0$, ie. le langage reconnue par notre automate:\\
$L_0=(a+b)^*a((a+b)^+a)^*b(a+b)^*$.\\


\end{document}
